\section{Test Case Generation Techiques}
\label{sec:background}
In this section, the processes of writing test cases manually and using automatic test case generation tools are discussed.  We also describe the automatic test case generation tools that are used in 

\subsection{Manually Written Tests}


\subsection{Automatically Generated Tests}
Due to the high cost and inconsistencies introduced when developing test suites by hand, automatic test suite generation research is on the rise.  In the past, the writing of test cases was left as an afterthought, and their generation was the responsibility of a separate quality assurance team rather than the developer.  This led to a disconnect between the code and the tests.  In recent years, however, there has been a move towards a more involved test development system in tandem with the the development process ~\cite{Gelperin:1988:GST:62959.62965}.  This movement includes focus on creating unit tests for code as its developed to ensure that code always passing tests thereby improving the quality of the code ~\cite{Canfora:2006:EAT:1159733.1159788}.  Although this improvement in test generation processes successfully increased the reliability of the code, the cost of time and effort to manually write high quality test cases increases as programs became more complex~\cite{clarke1998automated}. 

While many different techniques have been used to automatically generate tests, they can be divided into two key categories: Deterministic and Learning-Based.

\subsubsection{Deterministic}

CODEPRO  Given an input class, the tool creates a corresponding test class complete with multiple test methods for each input class method. The tool analyzes each method and input argument with the goal of generating test cases that exercise each line of code (the CodePro Code Coverage facility can provide feedback on how good your test cases are).

\cite{xie2009improving}

\subsubsection{Learning-Based}


Randoop~\cite{pacheco2007feedback} is another automatic unit test generator for Java that automatically creates unit tests for Java classes, in JUnit format.  Randoop generates unit tests using feedback-directed random test generation. This technique randomly generates sequences of methods and constructor invocations for the classes under test and uses
the sequences to create tests. Randoop then executes the sequences it creates and uses the results of the execution to create more assertions with the goal of avoiding redundant and illegal inputs while guiding towards generation of tests that lead to new object states. 

Evosuite