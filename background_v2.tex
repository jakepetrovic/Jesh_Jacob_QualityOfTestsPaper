\section{Background}
\label{sec:background}
In this section, the basic concepts behind automated test suite generation and mutation analysis are discussed as they relate to this paper.

\subsection{Automated Test Suite Generation}
Due to the high cost and inconsistencies introduced when developing test suites by hand, automatic test suite generation research is on the rise.  In the past, the writing of test cases was left as an afterthought, and their generation was the responsibility of a separate quality assurance team rather than the developer.  This led to a disconnect between the code and the tests.  In recent years, however, there has been a move towards a more involved test development system in tandem with the the development process ~\cite{Gelperin:1988:GST:62959.62965}.  This movement includes focus on creating unit tests for code as its developed to ensure that code always passing tests thereby improving the quality of the code ~\cite{Canfora:2006:EAT:1159733.1159788}.  Although this improvement in test generation processes successfully increased the reliability of the code, the cost of time and effort to manually write high quality test cases increases as programs became more complex~\cite{clarke1998automated}. 

To assist in the writing of test cases in an automated way, several types of tools have developed.  The simplest of these tools statically analyze the source code and create skeletons of  needed tests.  For example, JUnitDoclet \cite{JUnitDoclet} uses Javadoc to parse the source code of the application classes. From the collected information, JUnitDoclet writes TestCases and TestSuites where there is a TestSuite for each Java package, a TestCase for each public, non-abstract class, and a skeleton test method for each public method. In case of accessor tests, set and get are tested together.  The compiler will additionally guide the developer to challenging code segments such as classes that do not have a public constructor, classes that have no default constructors, and accessors for double or float values that need some epsilon when comparing two values.

While these test skeletons are helpful, more sophisticated tools have been developed that additionally analyze parameters and paths within the source code.  CoView~\cite{CoView}, for example, is a commercial Eclipse plug-in tool that uses its analyses of parameters and basic paths to create unit tests that are geared towards maximizing coverage with a minimal set of tests.  It strives to execute code logic and error conditions and allow developers to create easily and quickly as many what-if scenarios as desired. It also measures path and branch coverage at the project, package, class, method and test case level. 

CODEPRO  Given an input class, the tool creates a corresponding test class complete with multiple test methods for each input class method. The tool analyzes each method and input argument with the goal of generating test cases that exercise each line of code (the CodePro Code Coverage facility can provide feedback on how good your test cases are).

Randoop~\cite{pacheco2007feedback} is another automatic unit test generator for Java that automatically creates unit tests for Java classes, in JUnit format.  Randoop generates unit tests using feedback-directed random test generation. This technique randomly generates sequences of methods and constructor invocations for the classes under test and uses the sequences to create tests. Randoop then executes the sequences it creates and uses the results of the execution to create more assertions with the goal of avoiding redundant and illegal inputs while guiding towards generation of tests that lead to new object states. 

Finally, Evosuite~\cite{}
 