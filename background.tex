\section{Background}
\label{sec:background}
%----------------------------------------------------------------------------------------------------------------------------------------------------------------------------------------------------------------------------------

%----------------------------------------------------------------------------------------------------------------------------------------------------------------------------------------------------------------------------------
\subsection{Automated Test Suite Generation}
%----------------------------------------------------------------------------------------------------------------------------------------------------------------------------------------------------------------------------------
Software testing has evolved into a separate branch of software engineering, with a move towards a more involved test system involved in the development process ~\cite{Gelperin:1988:GST:62959.62965}. With the increase cost of time and effort, the idea of Automated tests suites began in the 1990's when Test Driven Design (TDD) began to draw the attention of software developers. TDD development advocated for a the more involved form of testing, with a focus on creating unit tests for code during development to insure the code was always passing tests thereby improving the quality of the code ~\cite{Canfora:2006:EAT:1159733.1159788}.  Although reliability of the code increased, the cost of time and effort to manually write the test cases increased as the programs became more complex over time ~\cite{clarke1998automated}. 

New tools arrived in he 2000's to mitigate this difficulty with the arrival of automated test generators. Unfortunately, these test suite generators were in an early and underdeveloped state upon their arrival, and were largely ignored by the software engineering industry. As time has passed, the automated test suite generators have matured much past their initial states.

Automated test suite generators are usually a plugin to an IDE or a command line tool. Tests are generated by evaluating the source code, and writing new test suite files that can be run to test the program.  Tests can be generated deterministically or non-deterministically. CodePro for example, generates the same set of tests for programs, whereas Evosuite may generate a slightly different test suite if ran multiple times. The deterministic nature of test suites can make them reliable and consistent, whereas the non-deterministic generation will seek to cover more edge cases in code, which may pose NFE incomplete problems. In the case of Evosuite, tests are evolve using Mutation-based assertion, incorporating a genetic algorithm to dynamically generate tests ~\cite{Fraser:2011:EAT:2025113.2025179}. 

%----------------------------------------------------------------------------------------------------------------------------------------------------------------------------------------------------------------------------------
\subsection{Mutation Analysis}
%----------------------------------------------------------------------------------------------------------------------------------------------------------------------------------------------------------------------------------
Mutation analysis is the process of evaluating the quality of test cases in a test suite. The objective is to identify test cases that either need to be added or updated to better test the code. This is done by injecting Artificial defects, or mutants, into the code ~\cite{Fraser:2010:MGU:1831708.1831728}.The test cases are then run against the fault-injected mutants. The mutants then expose errors in the test cases, and accordingly a mutation score is calculated. The mutation score represents the number of failed mutants to passing mutants. A higher mutation score is better because it indicates that the overall quality of the test cases are better. 

This research utilizes the mutation analysis tool MAJOR to receive a mutation score from each of the test suites. MAJOR executes mutation analysis in the java compiler, generting mutants into JUnit test cases with a conditional approach ~\cite{MAJOR:Just:2011}. An abstract syntax tree is generated to capture the original version of the program in the same basic block. MAJOR will execute the test suite, provide a mutation coverage, and then reorder the test cases based upon the execution time of test cases. 

%----------------------------------------------------------------------------------------------------------------------------------------------------------------------------------------------------------------------------------
\subsection{Genetic Algorithms}
%----------------------------------------------------------------------------------------------------------------------------------------------------------------------------------------------------------------------------------
Genetic algorithms are a subset of Evolutionary Algorithms ~\cite{Pandey:2012:GAC:2381716.2381766}.  Based off a data set, or population, the algorithm will select the most fit element represented by a binary string. Genetic Algorithms modify the population overtime to create a more fit set of elements. As the population evolves, the fitness of the elements increase. The goal is to output the most fit element. In complex problems, this may take extended periods of time, as it is a nondeterministic algorithm. Limits on fitness are often set to allow the algorithms to finish within a shorter time.

This technique is used in test suite generation in Evosuite to evolve the test cases. First, Evosuite generates a random set of tests to set up for mutation analysis.Then the tool uses a cover criterion based on branch coverage to create a fitness function for each test case ~\cite{Whole_Test_Suite_Generation:Fraser}. Finally, mutation analysis is used to evaluate the fitness of the test suite. The test suite dynamically evolves over time to improve the quality of the test. 